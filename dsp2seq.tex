%&latex
% Created from /l/mt/JOSOverview.tex

\newcommand{\theTitle}{Inventing Modern Sequence Models as a Music 320 Project}
\newcommand{\theSubTitle}{Samples become Meaning Vectors}
\newcommand{\theEvent}{CCRMA Open House} 
\newcommand{\theDate}{May 17, 2024}
% Classroom (Knoll 217), Friday 1:40-1:55pm}

\newcommand{\theAuthor}{Julius Smith}

%Mohonk05 said:
\input ../latex/stdpreshdr.tex % /w/latex/stdpreshdr.tex
\input ../latex/wgtmac.tex % /w/latex/wgtmac.tex

\usepackage{xcolor}
%N:\usepackage[dvipsnames]{xcolor}
%N:\usepackage[svgnames]{xcolor}

% Uses package etoolbox for \newtoggle et al:
\usepackage{etoolbox} % for \newtoggle et al
\newtoggle{local}
%\toggletrue{local}
\togglefalse{local}
\input localremote.tex

\date{\theDate}

\title{\theTitle}
\author{\theAuthor\\
  CCRMA Open House\\
%  Classroom (Knoll 217)\\
  Stanford University \\[10pt]
%  \theEvent
}

\begin{document}

\maketitle

%\section[\sectopts]{JOS Overview}

%\end{document}
%\endinput

\begin{slide}[\slideopts,toc={}]{Abstract}

Today's sequence models (such as large language models) in machine
learning (AI) arose from a blend of principle-based design and
empirical discovery, spanning several fields. This talk describes how
the ideas could have emerged from an elementary signal-processing
approach. This viewpoint offers some features:
\begin{enumerate}
\item Signal processing folks can quickly learn what is happening in a motivated way
\item Machine-learning experts might benefit from signal-processing insights
\item Obvious suggestions for things to try next naturally arise
\end{enumerate}

\href{https://ccrma.stanford.edu/ccrma-open-house}{[Open House Schedule]}

\end{slide}

\section[\sectopts,toc={Basic Idea}]{Music 320 Project Idea}

\begin{slide}[\slideopts,toc={One Pole Filter}]{One Pole Recursive Digital Filter}
\vspace{-2em}
  \myFigureToWidth{one-pole-jos}{0.6\twidth}{Pole at $z = -a_1$\\[5pt]
    $H(z) = \frac{\displaystyle Y(z)}{\displaystyle X(z)} = \frac{\displaystyle  b_0}{\displaystyle 1+a_1\,\zi}$}
\maybepause
\textbf{Idea: Let's Make an Associative Memory!}
\begin{itemize}
\mpitem $x(n)$ can be a \emph{long vector} $\xv(n)\in\RN$ representing \emph{anything we want} --- any ``label''
\mpitem Set $\beev_0 = 1$ and $\av_1 = -1$ to make $\yv(n)$ a \emph{sum of all input vectors} (``integrator'')
\mpitem Choose the dimension $N$ so large that \emph{vectors in the sum are mostly orthogonal}
\mpitem Retrieve similar vectors using a \emph{matched inner product} $\wv^T\xv > b$,\\
        for some suitable threshold $b$ (Hey! That's one simulated neuron! (``Perceptron''))
\end{itemize}
\end{slide}

\begin{slide}[\slideopts,toc={Inner Product}]{Vector Retrieval by Inner Product}

  Given the sum of vectors
  \[
  \yv(n) = \sum_{m=0}^n \xv(m)
  \]
  and a ``query vector'' $\wv = \xv(k)$,\\
  \maybepause
  find the query in the sum using an \emph{inner product:}
  \[
  \wv^T\yv(n) \eqsp \sum_{m=0}^n \wv^T\xv(m) \;\approx\; \xv^T(k)\,\xv(k) \eqsp \|\xv(k)\|^2 \;>\; b(k)
  \]
  where $b(k)$ is the \emph{detection threshold} for $\xv(k)$

  \begin{itemize}
  \mpitem This works because the spatial dimension is so large that $\xv^T(j)\,\xv(k)\approx 0$ for $j\ne k$

  \mpitem Retrieval threshold $b(k)$ depends on $\|\xv(k)\|^2$\\
  $\Rightarrow$ \emph{reserve the radial dimension for similarity scoring}

  \mpitem \Ie, \emph{only populate the \textbf{hypersphere}} in $\RN$: $\norm{\xv(k)} = 1, \, \forall k$

  \mpitem We just invented \textbf{\texttt{RMSNorm}}, used extensively in neural networks (not \texttt{LayerNorm})

  \end{itemize}

\end{slide}

\begin{slide}[\slideopts,toc={Vector Memory}]{Cumulative Vector Memory}
\vspace{-1em}
\myFigureToWidth{integrator}{0.6\twidth}{}
\vspace{-1em}
\myFigureToWidth{ginaSphere1}{0.5\twidth}{}
%\myTwoFiguresToWidth{integrator}{ginaSphere1}{0.5\twidth}{Vector summer}{MidJourney depiction for $N=3$}{}
\end{slide}

\begin{slide}[\slideopts,toc={Gating}]{Gated Vector Memory}

\myFigureToWidth{integrator}{0.6\twidth}{Input Vector Summer}

\begin{itemize}
\item \textbf{Problem:} Need a \emph{memory reset}
\mpitem \textbf{Solution:} Set \emph{feedback gain to zero} for one step to clear the memory
\item[]
\mpitem \textbf{Problem:} Need an \emph{input gate} to suppress unimportant inputs
\mpitem \textbf{Solution:} Set \emph{input gain to zero} for unimportant inputs
\item[]
\mpitem We just invented \textbf{gating}, used extensively in neural sequence models
\end{itemize}

\end{slide}

\begin{slide}[\slideopts,toc={Gated RNN}]{Gated Recurrent Network}

\vspace{-1em}

  \textbf{Idea:} \emph{Learn} the input and feedback gates as functions of $\xv_n$\\
  based on many input-output examples $(\xv_n,\yv_n)$ (``training data''):
%% \mpitem Some item
%% \end{itemize}

\vspace{-1em}

\myFigureToWidth{one-pole-rnn}{0.5\twidth}{Vector Memory with Learned Input and Feedback Gates}

\maybepause

\textbf{``Obvious'' Training Considerations:}
\begin{itemize}
\mpitem Initialize $-\av_1$ for desired initial memory duration (exponentially fading)
\mpitem Learn $-\av_1(\xv_n)$ as $\Imtx\cdot e^{-\Delta}\approx \Imtx -\Imtx \Delta$, and $\bv_0(\xv_n)$ as $\mbox{Linear}(\xv_n,\yv_n)\cdot\Delta$, where
$\Delta = \mbox{softPlus}(\mbox{parameter}(\xv_n,\Yv_n))$ (approximately gain-normalized)
\end{itemize}

\end{slide}

\begin{slide}[\slideopts,toc={Skip Connection}]{Output Gating}

  \vspace{-1em}

  \textbf{Idea:} Since we have input and feedback gates, why not an \textbf{\emph{output gate?}}

\myFigureToWidth{one-pole-rnn-skip}{0.6\twidth}{Gated RNN with \textbf{Skip Connection}}

\maybepause
Output gating allows network to be ``bypassed'' when not helpful

\maybepause
\vspace{1em}
\textbf{Idea:} For detecting vectors in $\yv$ using \emph{learned} inner-product weights and thresholds,\\
use an array of \emph{``Perceptrons''} (P)
%\emph{``Multi-Layer Perceptron'' (MLP)}
\begin{itemize}
\mpitem Each P detects one or more memory vectors similar to its weight vectors
\mpitem The P outputs indicate \emph{which weight-vectors are present} in the vector sum
\end{itemize}

\end{slide}

\begin{slide}[\slideopts,toc={Sequences}]{Sequence Modeling}
  \vspace{-1em}
  \begin{itemize}
    \mpitem If each vector represents a \emph{word,} a vector sum is simply a \emph{bag of words}
    \mpitem To model a \emph{sequence} of words, we have options:
    \begin{enumerate}
      \mpitem Use a \emph{positional encoding} scheme, such as
      \begin{enumerate}
        \mpitem \emph{Amplitude Decay} - Multiply the sum by a \emph{forgetting factor} each sequence step\\
        (RNNs) - \emph{poor choice} (conflates with angular distance on the hypersphere)
        \mpitem \emph{Sinusoidal Amplitude Modulation} - Add a sinusoid with \emph{increasing frequency} to each vector summing into the history\\
        (used in the original Transformer)
        \mpitem \emph{Phase Shift} - Multiply by the sum by $e^{j\Delta}$ each sample\\
        (``RoPE'') - \emph{apparently most used today}
      \end{enumerate}
      \mpitem Use many vector-sum memories in parallel, positionally encoded\\
      (``State Expansion'' in SSMs)
      % \mpitem Learn position-specific \emph{FIR Coefficients} across a parallel bank of positionally encoded vector-sum memories (``Attention Layer'' in a Transformer)
    \end{enumerate}
  \end{itemize}
\end{slide}

\begin{slide}[\slideopts]{State Expansion}

% \vspace{-4em}
% \myFigureToWidth{one-pole-rnn-seq}{0.6\twidth}{} % {Sequence Modeling with Gated RNNs}
%

\vspace{-2em}
\myFigureToWidth{statespace-rnn}{0.6\twidth}{}
\vspace{-2em}
\begin{itemize}
  \mpitem Called a \emph{``Structured State-space Model''} (SSM) in machine learning
  \mpitem Feedback matrix $\Amtx$ has been \emph{diagonal} since ``S4D'' (2022)\\
  	  $\Rightarrow$ Parallel bank of vector one-poles (\emph{gated, state-expanded RNNs})
  \mpitem Processed sequence (``context buffer'') is \emph{indefinitely long}
  \mpitem Gating matrices are typically simple linear input projections, \eg,
  \[
    [\Bmtx(\xv_n), \Cmtx(\xv_n)] = \Lmtx\,\xv_n  % , \qquad \Dmtx(\xv_n) = \mbox{SiLU}(\Lmtx^\prime\xv_n)\]
  \] 
  (See Mamba, \eg)
  % \mpitem Conv1D mixing on input
\end{itemize}

\end{slide}

\begin{slide}[\slideopts,toc={Attention}]{Attention Layer}

\vspace{-1em}

\textbf{Idea:} Also use \emph{FIR Filtering}

%\myFigureToWidth{one-pole-rnn-seq}{0.6\twidth}{} % {Sequence Modeling with Gated RNNs}
%\vspace{-4em}

\myFigureToWidth{attention}{0.8\twidth}{}

\vspace{-1em}

We can have \emph{separately learned FIR coefficient matrices} $b_j[\xv(n-j),k]$, \\
  which depend on the
  \begin{enumerate}
    \mpitem input \emph{position} $j$ in the input sequence (``context buffer'')
    \mpitem input \emph{vector} $\xv(n-j)$, $j=0,1,2,\dots,M$, (nonlinear)
    \mpitem \emph{output-position} $k$ being computed, $k=0,1,2,\dots,M$ ($M+1$ outputs)
  \end{enumerate}

\maybepause
\textbf{Idea:} Add \emph{relevance gating} suppressing unimportant inputs to each output (``attention'')

\vspace{0.5em}

\maybepause
\textbf{Idea:} Measure relevance using an \emph{inner product} between the output and input positions (``dot-product attention'')
\end{slide}

\begin{slide}[\slideopts]{Dot-Product Attention}

\vspace{-1em}

\myFigureToWidth{attention}{0.8\twidth}{}

\vspace{-1em}

\textbf{Relevance Gating}

Contribution from input $\xv(n-j)$ to the nonlinear FIR sum for output $\yv(n-k)$ is $(Q_k^T K_j)\xv(n-j)$, where
\begin{itemize}
  \mpitem $Q_k[\xv(n-k),k]$ is called the \emph{query} vector for position $k$ in the input sequence
  \mpitem $K_j[\xv(n-j),j]$ is called the \emph{key} vector for position $j$ in the input sequence
%  \mpitem $V_j[\xv(n-j)]$ is called the \emph{value} vector for position $j$ in the input sequence
\end{itemize}

\maybepause

\textbf{Idea:} To provide more flexibility for the attention sum,
replace $\xv(n-j)$ in the attention sum with a learned \emph{value
vector} $V_j[\xv(n-j)]$\\
%
\maybepause
$\Rightarrow$\\
%
Relevance of input $j$ to output $k$ $\propto (Q_k^T K_j)V_j$, where
all three vectors are learned projections of $\xv(n-j)$ for each
$(j,k)$ (``Transformer'')

\end{slide}

\begin{slide}[\slideopts]{Multi-Head Attention}

\textbf{Idea:} To support multiple meaning possibilities, \emph{partition the model space} into\\
parallel independent \emph{attention calculations} (``multi-head attention'')

%\myFigureToWidth{one-pole-rnn-seq}{0.6\twidth}{} % {Sequence Modeling with Gated RNNs}
%\vspace{-4em}

\begin{itemize}
  \mpitem Each \emph{attention head} can form an independent sequence model
  \mpitem Also introduced in the Transformer paper (2017)
\end{itemize}

\end{slide}

\begin{slide}[\slideopts]{Weight Tying}

  When the sequence model maps input to output in the same ``language'' (e.g., English to English),
  it makes sense to use the \emph{same embedding vectors} at the input and output layers, instead
  of separately learning a set of weights for mapping to the final output.
  This is called ``weight tying'' (many fewer parameters, better results).

\end{slide}

\begin{slide}[\slideopts]{Hierarchical Blocks}
\begin{itemize}
  \mpitem Cascade blocks of attention + MLP and/or gated recurrence + MLP to model \emph{hierarchical relationships} like image features or grammatical constructs
  \mpitem Attention and gated RNNs are called ``mixing layers'' (successive inputs are combined)
  \mpitem MLPs are called ``point transformations'' (generally mapping of any vector from one place to another)
  \mpitem RMSNorm typical at the input to put it on the hypersphere --- also used internally (see Hawk/Griffin e.g.)
\end{itemize}
\end{slide}

%% \begin{slide}[\slideopts,toc={Multiple Heads}]{MultiHead Attention}

%% \vspace{-1em}

%% \textbf{Idea:} To enable several independent meaning possibilities, \emph{partition the space}

%% %\myFigureToWidth{one-pole-rnn-seq}{0.6\twidth}{} % {Sequence Modeling with Gated RNNs}
%% %\vspace{-4em}

%% \begin{itemize}
%%   \mpitem Each \emph{head} can form an independent sequence model
%%   \mpitem Filter coefficients computed from input as before, but smaller
%%   \mpitem Part of the \emph{Transformer} for sequence modeling (attention layer)
%% \end{itemize}

%% \end{slide}

%% \begin{slide}[\slideopts,toc={Other Features}]{Other ``Obvious'' Features}
%% \begin{itemize}
%%   \mpitem Tying output ``language modeling head'' weights to the input embedding weights
%%   \mpitem Positional encoding within an RNN
%% \end{itemize}
%% Less obvious:
%% \begin{itemize}
%%   \mpitem Multihead Attention (down-projection + independent spatial processing)
%%   \mpitem (Q,K,V) matrices for ``dot-product attention'' (down-projection + inner-product function + value flexibility)
%% \end{itemize}
%% \end{slide}

%% \begin{wideslidewhite}[\slideopts,toc={Architectures}]{Architectures}
%% \vspace{-4em}
%% \myFigureToWidth{Architectures}{\twidth}{}
%% \vspace{-4em}
%% \end{wideslidewhite}

\begin{slide}[\slideopts,toc={Hypersphere}]{Where Meaning Lies}
\vspace{-1em}
\myFigureToWidth{ginaSphere1}{\twidth}{}
\end{slide}

\section[\sectopts,toc={History Samples}]{Sequence Modeling Snapshots}

\begin{slidewhite}[\slideopts, toc={LSTM \& GRU}]{LSTM and GRU}
\vspace{-6em}
\myFigureRotateToWidth{Architectures1}{-90}{\twidth}{}
\end{slidewhite}

\begin{slidewhite}[\slideopts, toc={SSM \& Mamba}]{Structured State Space and Mamba}
\vspace{-6em}
\myFigureRotateToWidth{Architectures2}{-90}{\twidth}{}
\end{slidewhite}

\begin{slidewhite}[\slideopts,toc={Hawk \& Griffin}]{Hawk and Griffin}
\vspace{-6em}
\myFigureRotateToWidth{Architectures3}{-90}{\twidth}{}
\end{slidewhite}

\begin{slidewhite}[\slideopts,toc={HGRN2}]{Gated Linear RNNs with State Expansion}
\vspace{-6em}
\myFigureRotateToWidth{Architectures4}{-90}{\twidth}{}
\end{slidewhite}

\begin{slidewhite}[\slideopts,toc={RWKV+}]{RWKV, Eagle, Finch}
\vspace{-6em}
\myFigureRotateToWidth{Architectures5}{-90}{\twidth}{}
\end{slidewhite}

%\begin{slidewhite}[\slideopts,toc={}]{Architectures 6}
%\vspace{-6em}
%\myFigureRotateToWidth{Architectures6}{-90}{\twidth}{}
%\end{slidewhite}

%\input ai-reading-2022.tex
%\input ai-reading-2023.tex
%\input ai-projects.tex
%\section[\sectopts]{Differentiable DSP (DDSP)}
%\input ddsp.tex

\end{document}
\endinput
