\begin{slide}[\slideopts,method=direct,toc={}]{Overview}

Sequence Models, such as Large Language Models (LLMs), arose from a blend of principle-based design and empirical discovery, spanning several fields.
\maybepause
This talk describes how the ideas could have emerged from an elementary signal-processing approach.
\maybepause
This viewpoint offers some features:
\begin{enumerate}
\mpitem Signal processing folks can quickly learn what is happening in a motivated way
\mpitem Machine-learning experts might benefit from signal-processing insights
\mpitem Obvious suggestions for things to try next naturally arise
\end{enumerate}

% \href{https://ccrma.stanford.edu/ccrma-open-house}{[Open House Schedule]}

\vspace{1em}
\maybepause
\textbf{Plan:}
%\begin{enumerate}
\maybepause
%\mpitem
``Invent'' the components of modern sequence models from basic signal processing
%\end{enumerate}

\vspace{1em}
\maybepause
% Requires method=direct:
\textbf{Overheads and more:} \href{https://ccrma.stanford.edu/~jos/Welcome.html#dsponline24}{https://ccrma.stanford.edu/$\sim$jos/Welcome.html\#dsponline24}
%\textbf{Overheads and more:} \href{\url{https://ccrma.stanford.edu/~jos/Welcome.html#dsponline24}}{foo} % https://ccrma.stanford.edu/\~{}jos/Welcome.html\#dsponline24}

\end{slide}

%% \begin{slide}[\slideopts]{Intended Audience}
%% \maybepause
%% Precursors to this talk were aimed at
%% \begin{enumerate}
%% \mpitem CCRMA students after one introductory signal-processing course
%% \mpitem West Coast Machine Learning meetup group: mostly in machine learning
%% \end{enumerate}
%% \maybepause
%% This version is aimed at
%% \begin{enumerate}
%%   \mpitem[3.] Signal-processing students and professionals,\\
%%               assumed to know more DSP than machine learning (like me)
%% \end{enumerate}
%% \end{slide}
